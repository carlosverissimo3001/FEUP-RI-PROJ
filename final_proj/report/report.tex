\documentclass[conference]{IEEEtran}
\IEEEoverridecommandlockouts
% The preceding line is only needed to identify funding in the first footnote. If that is unneeded, please comment it out.
\usepackage{cite}
\usepackage{amsmath,amssymb,amsfonts}
\usepackage{algorithmic}
\usepackage{hyperref}
\usepackage{graphicx}
\usepackage{textcomp}
\usepackage{xcolor}
\def\BibTeX{{\rm B\kern-.05em{\sc i\kern-.025em b}\kern-.08em
    T\kern-.1667em\lower.7ex\hbox{E}\kern-.125emX}}
\begin{document}

\title{Humanoid Sprinting and Stopping DRL\\
}

\author{

    \IEEEauthorblockN{Carlos Veríssimo}
    \IEEEauthorblockA{\textit{Department of Informatics Engineering} \\
        \textit{FEUP}\\
        Porto, Portugal \\
        up201907716@up.pt }
    \and

    \IEEEauthorblockN{Miguel Amorim}
    \IEEEauthorblockA{\textit{Department of Informatics Engineering } \\
        \textit{FEUP}\\
        Porto, Portugal \\
        up201907756@up.pt }
    \and

    \IEEEauthorblockN{Rafael Camelo}
    \IEEEauthorblockA{\textit{Department of Informatics Engineering } \\
        \textit{FEUP}\\
        Porto, Portugal \\
        up201907729@up.pt }
}


\maketitle

\begin{abstract}



\end{abstract}

\begin{IEEEkeywords}

    RoboCup, RoboCup 3D Simulation League, Reinforcement Learning, Humanoid Sprinting, Humanoid Stopping, Soccer

\end{IEEEkeywords}

\section{Introduction}

Reinforcement Learning (RL) is a machine learning field in which agents learn optimal behaviors through trial-and-error
interactions with their environment, with the goal of maximizing rewards and minimizing punishments.

The RoboCup, which began in 1997, is a well-known platform for promoting robots and AI by pushing participants with tasks such
as soccer, rescue, and home care. \cite{robocup97}.

This article digs into the RoboCup's 3D Simulation League, introduced in 2004. The league evolved from using simple spheres to humanoid robots and
now uses a simulated version of the NAO robot, a 58-cm-tall robot with 25 degrees of freedom (DOF) \cite{naorobot}, as the robot model.

To enforece physics laws, coordinate communications between the server and clients and to officiate the game, the league uses the SimSpark simulator.

Sprinting is an important feature of robotic soccer and a key role in team success. This paper also focuses on the difficult
problem of stopping a run without falling, a feat that necessitates detailed control over the robot's numerous joints and sensors.

This study focuses on improving humanoid robot running and halting in the RoboCup 3D Simulation League.
Recognizing the intricacy and significance of these talents in robotic soccer, we intend to expand on previous advances in the field.
Our primary goal is to create solid, reliable techniques for accelerating and decelerating without falling, by making use of the comprehensive control
of the robot's joints and sensors.

Subsequent sections detail related works (Section \ref{Related Work}), our methodology (Section \ref{Methodology}),
the significant results and their implications (Section \ref{Results and Discussion}),
concluding with future directions for this field (Section \ref{Conclusions and Future Work})

\section{Related Work}\label{Related Work}


\section{Methodology}\label{Methodology}

\section{Results and Discussion}\label{Results and Discussion}

\section{Conclusions and Future Work}\label{Conclusions and Future Work}

\section{Acknowledgments}\label{Acknowledgments}


\begin{thebibliography}{00}
    \bibitem{lau2023fc}Lau, N., Reis, L., Simoes, D., Abreu, M., Silva, T. \& Resende, F. FC Portugal 3D Simulation Team: Team Description Paper 2020.  (2023)
    \bibitem{10.1007/978-3-030-35699-6_1}Abreu, M., Reis, L. \& Lau, N. Learning to Run Faster in a Humanoid Robot Soccer Environment Through Reinforcement Learning. {\em RoboCup 2019: Robot World Cup XXIII}. pp. 3-15 (2019)
    \bibitem{10.1007/978-3-030-36150-1_44}Simões, D., Amaro, P., Silva, T., Lau, N. \& Reis, L. Learning Low-Level Behaviors and High-Level Strategies in Humanoid Soccer. {\em Robot 2019: Fourth Iberian Robotics Conference}. pp. 537-548 (2020)
    \bibitem{naorobot} Nao the Humanoid and Programmable Robot, \href{www.aldebaran.com/en/nao}{www.aldebaran.com/en/nao}. Last accessed 24 Dec 2023
    \bibitem{robocup97} Noda, I., Suzuki, S. J., Matsubara, H., Asada, M., Kitano, H.: RoboCup-97: The first robot world cup soccer games and conferences. AI magazine 19(3), 49–49 (1998)
\end{thebibliography}

\end{document}
