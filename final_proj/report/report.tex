\documentclass[conference]{IEEEtran}
\IEEEoverridecommandlockouts
% The preceding line is only needed to identify funding in the first footnote. If that is unneeded, please comment it out.
\usepackage{cite}
\usepackage{amsmath,amssymb,amsfonts}
\usepackage{algorithmic}
\usepackage{graphicx}
\usepackage{textcomp}
\usepackage{hyperref}
\usepackage{xcolor}
\usepackage{float}

\usepackage{tabularx,booktabs}
\newcolumntype{C}{>{\centering\arraybackslash}X}
\setlength{\extrarowheight}{1pt} % for a bit more open "look"
\usepackage{graphicx}

\def\BibTeX{{\rm B\kern-.05em{\sc i\kern-.025em b}\kern-.08em
    T\kern-.1667em\lower.7ex\hbox{E}\kern-.125emX}}
\begin{document}

\title{Humanoid Sprinting and Stopping DRL\\
}

\author{

    \IEEEauthorblockN{Carlos Veríssimo}
    \IEEEauthorblockA{\textit{Department of Informatics Engineering} \\
        \textit{FEUP}\\
        Porto, Portugal \\
        up201907716@up.pt }
    \and

    \IEEEauthorblockN{Miguel Amorim}
    \IEEEauthorblockA{\textit{Department of Informatics Engineering } \\
        \textit{FEUP}\\
        Porto, Portugal \\
        up201907756@up.pt }
    \and

    \IEEEauthorblockN{Rafael Camelo}
    \IEEEauthorblockA{\textit{Department of Informatics Engineering } \\
        \textit{FEUP}\\
        Porto, Portugal \\
        up201907729@up.pt }
}


\maketitle

\begin{abstract}



\end{abstract}

\begin{IEEEkeywords}

RoboCup, RoboCup 3D Simulation League, Deep Reinforcement Learning, Humanoid Sprinting, Humanoid Stopping

\end{IEEEkeywords}

\section{Introduction}

The RoboCup is an international robotics competition founded in 1997 with the main goal of promoting robotics and AI research,
by providing a standard problem where a wide range of technologies can be integrated and examined.
The RoboCup Federation organizes a variety of competitions, including soccer, rescue, and home assistance.
The soccer competition is the most popular one, and it is divided into several leagues.

This research paper focuses on the Simulation League in which independent software agents play on a virtual soccer
field, more specifically on the 3D simulation league, where the agents are based on the NAO humanoid robot.

Sprinting is a fundamental skill in soccer, and it is a key factor in the success of a team as it allows
the team to quickly move the ball towards the opponent's goal.

Reinforcement Learning (RL) is a machine learning technique that allows an agent to learn by
interacting with its environment. The agent learns to achieve a goal in an uncertain,
potentially complex environment. Agents get rewarded for performing correct actions and punished for
performing incorrect ones.

The main goal of this research paper is to develop a Deep Reinforcement Learning (DRL) agent
that is able to sprint and stop in a simulated soccer environment.


\section{Related Work}

\section{Methodology}

\section{Results and Discussion}

\section{Conclusions and Future Work}

\section{Acknowledgments}


%\begin{thebibliography}{00}
%\bibitem{b1} G. Eason, B. Noble, and I. N. Sneddon, ``On certain integrals of Lipschitz-Hankel type involving products of Bessel functions,'' Phil. Trans. Roy. Soc. London, vol. A247, pp. 529--551, April 1955.
%\bibitem{b2} J. Clerk Maxwell, A Treatise on Electricity and Magnetism, 3rd ed., vol. 2. Oxford: Clarendon, 1892, pp.68--73.
%\bibitem{b3} I. S. Jacobs and C. P. Bean, ``Fine particles, thin films and exchange anisotropy,'' in Magnetism, vol. III, G. T. Rado and H. Suhl, Eds. New York: Academic, 1963, pp. 271--350.
%\bibitem{b4} K. Elissa, ``Title of paper if known,'' unpublished.
%\bibitem{b5} R. Nicole, ``Title of paper with only first word capitalized,'' J. Name Stand. Abbrev., in press.
%\bibitem{b6} Y. Yorozu, M. Hirano, K. Oka, and Y. Tagawa, ``Electron spectroscopy studies on magneto-optical media and plastic substrate interface,'' IEEE Transl. J. Magn. Japan, vol. 2, pp. 740--741, August 1987 [Digests 9th Annual Conf. Magnetics Japan, p. 301, 1982].
%\bibitem{b7} M. Young, The Technical Writer's Handbook. Mill Valley, CA: University Science, 1989.
%\end{thebibliography}

\appendices

\end{document}
